\documentclass[11pt]{amsart}
\usepackage[round]{natbib}
\usepackage{bibentry}
\renewcommand{\baselinestretch}{1.5}

\newcommand{\ba}{{\bf a}}
\newcommand{\bb}{{\bf b}}
\newcommand{\ta}{{\tilde a}}
\newcommand{\tb}{{\tilde b}}
\newcommand{\bab}{\bar{{\bf a}}}
\newcommand{\bbb}{\bar{{\bf b}}}
\newcommand{\byb}{\bar{{\bf y}}}
\newcommand{\be}{{\bf e}}
\newcommand{\bu}{{\bf u}}
\newcommand{\bk}{{\bf k}}
\newcommand{\bp}{{\bf p}}
\newcommand{\bq}{{\bf q}}
\newcommand{\bt}{{\bf t}}
\newcommand{\bx}{{\bf x}}
\newcommand{\by}{{\bf y}}
\newcommand{\bz}{{\bf z}}
\newcommand{\bv}{{\bf v}}
\newcommand{\bw}{{\bf w}}
\newcommand{\bg}{{\bg b}}
\newcommand{\bs}{{\bf s}}
\newcommand{\bI}{{\bf I}}
\newcommand{\bR}{{\bf R}}
\newcommand{\bT}{{\bf T}}
\newcommand{\binf}{\mbox{\boldmath $\infty$}}
\newcommand{\thh}{\mbox{\boldmath $\theta$}}
\newcommand{\la}{\mbox{\boldmath $\lambda$}}
\newcommand{\bbt}{\mbox{\boldmath $\beta$}}
\newcommand{\muu}{\mbox{\boldmath $\mu$}}
\newcommand{\brho}{\mbox{\boldmath $\rho$}}
\newcommand{\bdel}{\mbox{\boldmath $\delta$}}
\newcommand{\Sig}{\mbox{\boldmath $\Sigma$}}
\newcommand{\Ph}{\mbox{\boldmath $\Phi$}}
\newcommand{\beps}{\mbox{\boldmath $\epsilon$}}
\newcommand{\bbet}{\mbox{\boldmath $\beta$}}



\begin{document}

\title{On Multivariate $t$ and Gau{\ss} Probabilities in R}

\author{Torsten Hothorn}
\address{Friedrich-Alexander-Universit\"at Erlangen-N\"urnberg \\
Institut f\"ur Medizininformatik, Biometrie und Epidemiologie \\
Waldstra{\ss}e 6, D-91054 Erlangen}
\email{Torsten.Hothorn@rzmail.uni-erlangen.de}
\author{Frank Bretz}
\address{Universit\"at Hannover \\ LG Bioinformatik, FB Gartenbau \\
Herrenh\"auser Str. 2 \\ D-30419 Hannover}
\email{bretz@ifgb.uni-hannover.de}
\author{Alan Genz}
\address{Department of Mathematics \\ Washington State University \\
Pullman, WA 99164-3113 USA}
\email{alangenz@wsu.edu}

\maketitle

\section{Algorithms}

The numerical computation if a multivariate normal or $t$ probability is
often a difficult problem. Recent developments resulted in algorithms for
the fast computation of those probabilities for arbitrary correlation
structures. We refer to the work described in \cite{numerical-:1992},
\cite{comparison:1993} and \cite{numerical-:1999}. The procedures proposed
in those papers are implemented in package {\ttfamily mvtnorm}, available at
CRAN. We first illustrate the use of the package using a simple example.

\section{A Simple Example}

Assume that $ X = (X_1, X_2, X_3) $ is multivariate normal with correlation
matrix
\begin{eqnarray*}
\Sigma = \left( \begin{array}{ccc} 1 & \frac{3}{5} & \frac{1}{3} \\
\frac{3}{5} & 1 & \frac{11}{15} \\
\frac{1}{3} & \frac{11}{15} & 1 \end{array} \right)
\end{eqnarray*}
and expectation $ \mu = 0 $. We are interested in the probability 
\begin{eqnarray*}
P(-\infty < X_1 \le 1, -\infty < X_2 \le 4, -\infty < X_3 \le 2). 
\end{eqnarray*}
This is computed as follows:
\begin{verbatim}
R> m <- 3
R> sigma <- diag(3)
R> sigma[2,1] <- 3/5
R> sigma[3,1] <- 1/3
R> sigma[3,2] <- 11/15
R> pmvnorm(mean=rep(0, m), sigma, lower=rep(-Inf, m), upper=c(1,4,2))
$value
[1] 0.8279846

$error
[1] 2.696757e-07

$msg
[1] "Normal Completion"
\end{verbatim}
First, the lower triangular of the correlation matrix is needed. The mean
vector is passed to pmvnorm by the argument {\ttfamily mean}. The region of
integration is given by {\ttfamily lower} and {\ttfamily upper}, both can
take {\ttfamily -Inf} or {\ttfamily +Inf}. The value of {\ttfamily pmvnorm}
is a list with the following components:
\begin{itemize}
\item {\ttfamily value}: the estimated integral value,
\item {\ttfamily error}: the estimated absolute error,
\item {\ttfamily msg}: a status message, indicating wheater or not the algorithm
terminated correctly.
\end{itemize}
From the results above it follows that
\begin{eqnarray*}
P(-\infty < X_1 \le 1, -\infty < X_2 \le 4, -\infty < X_3 \le 2) \approx
0.82798
\end{eqnarray*}
with an absolute error of $2.7e-07$. 

\section{Details}

This section outlines the basic ideas of the algorithms used. The
multivariate $t$ distribution (MVT) is given by $$ \bT(\ba, \bb,
\Sig, \nu) = \frac{2^{1-\frac{\nu}{2}}}{\Gamma(\frac{\nu}{2})}
\int\limits_0^{\infty}s^{\nu-1}e^{-\frac{s^2}{2}}
\Ph(\frac{s\ba}{\sqrt{\nu}},\frac{s\bb}{\sqrt{\nu}},\Sig) ds, $$
where the multivariate normal distribution function (MVN) $$
\Ph(\ba,\bb, \Sig) = \frac{1}{\sqrt{|\Sig| (2\pi)^m}}
\int\limits_{a_1}^{b_1} \int\limits_{a_2}^{b_2} ...
\int\limits_{a_m}^{b_m} e^{- \frac{1}{2} \bx^t \Sig^{-1} \bx}
d\bx, $$ $\bx = (x_1, x_2, ..., x_m)^t$, $-\infty \leq a_i < b_i
\leq \infty$ for all $i$, and $\Sig$ is a positive semi-definite
symmetric $m \times m$ matrix. The original integral over an
arbitrary $m$-dimensional, possibly unbounded hyper-rectangle is
transformed to an integral over the unit hypercube. These
transformations are described e.g. in \cite{numerical-:1992} for the MVN
case and in \cite{numerical-:1999} for the MVT case. Several
suitable standard integration routines can be applied to this
transformed integral. For the present implementation randomized
lattice rules were used. Such lattice rules seek to fill the
integration region evenly in a deterministic process. In
principle, they construct regular patterns, such that the
projections of the integration points onto each axis produce an
equidistant subdivision of the axis. Robust integration error
bounds are obtained by introducing additional shifts of the
entire set of integration nodes in random directions. Since this
additional randomization step is only performed to introduce a
robust Monte Carlo error bound, 10 simulation runs are usually
sufficient. For a more detailed description \cite{numerical-:1999}
might be referred to.

\section{Applications}

Here we should give a 'real life' example (multiple testing or something
like that). Maybe we can bring someone to the idea to hack this stuff ;-)

\bibliographystyle{plainnat}
\bibliography{litdb}

\end{document}
