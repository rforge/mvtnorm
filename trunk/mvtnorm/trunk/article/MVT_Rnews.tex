\documentclass[11pt]{amsart}
\usepackage[round]{natbib}
\usepackage{bibentry}
\renewcommand{\baselinestretch}{1.5}

\begin{document}

\title{On Multivariate $t$ and Gau{\ss} Probabilities in R}

\author{Torsten Hothorn}
\address{Friedrich-Alexander-Universit\"at Erlangen-N\"urnberg \\
Institut f\"ur Medizininformatik, Biometrie und Epidemiologie \\
Waldstra{\ss}e 6, D-91054 Erlangen}
\email{Torsten.Hothorn@rzmail.uni-erlangen.de}
\author{Frank Bretz}
\address{Universit\"at Hannover \\ LG Bioinformatik, FB Gartenbau \\
Herrenh\"auser Str. 2 \\ D-30419 Hannover}
\email{bretz@ifgb.uni-hannover.de}
\author{Alan Genz}
\address{Department of Mathematics \\ Washington State University \\
Pullman, WA 99164-3113 USA}
\email{alangenz@wsu.edu}

\maketitle

\section{Algorithms}

The numerical computation if a multivariate normal or $t$ probability is
often a difficult problem. Recent developments resulted in algorithms for
the fast computation of those probabilities for arbitrary correlation
structures. We refer to the work described in \cite{numerical-:1992},
\cite{comparison:1993} and \cite{numerical-:1999}. The procedures proposed
in those papers are implemented in package {\ttfamily mvtnorm}, available at
CRAN. We first illustrate the use of the package using a simple example.

\section{A Simple Example}

Assume that $ X = (X_1, X_2, X_3) $ is multivariate normal with correlation
matrix
\begin{eqnarray*}
\Sigma = \left( \begin{array}{ccc} 1 & \frac{3}{5} & \frac{1}{3} \\
\frac{3}{5} & 1 & \frac{11}{15} \\
\frac{1}{3} & \frac{11}{15} & 1 \end{array} \right)
\end{eqnarray*}
and expectation $ \mu = 0 $. We are interested in the probability 
\begin{eqnarray*}
P(-\infty < X_1 \le 1, -\infty < X_2 \le 4, -\infty < X_3 \le 2). 
\end{eqnarray*}
This is computed as follows:
\begin{verbatim}
R> m <- 3
R> sigma <- diag(3)
R> sigma[2,1] <- 3/5
R> sigma[3,1] <- 1/3
R> sigma[3,2] <- 11/15
R> pmvnorm(mean=rep(0, m), sigma, lower=rep(-Inf, m), upper=c(1,4,2))
$value
[1] 0.8279846

$error
[1] 2.696757e-07

$msg
[1] "Normal Completion"
\end{verbatim}
First, the lower triangular of the correlation matrix is needed. The mean
vector is passed to pmvnorm by the argument {\ttfamily mean}. The region of
integration is given by {\ttfamily lower} and {\ttfamily upper}, both can
take {\ttfamily -Inf} or {\ttfamily +Inf}. The value of {\ttfamily pmvnorm}
is a list with the following components:
\begin{itemize}
\item {\ttfamily value}: the estimated integral value,
\item {\ttfamily error}: the estimated absolute error,
\item {\ttfamily msg}: a status message, indicating wheater or not the algorithm
terminated correctly.
\end{itemize}
From the results above it follows that
\begin{eqnarray*}
P(-\infty < X_1 \le 1, -\infty < X_2 \le 4, -\infty < X_3 \le 2) \approx
0.82798
\end{eqnarray*}
with an absolute error of $2.7e-07$. 

\section{Details}

This section outlines the basic ideas of the algorithms used (short!
non-technical) ;-) Frank?

\section{Applications}

Here we should give a 'real life' example (multiple testing or something
like that). Maybe we can bring someone to the idea to hack this stuff ;-)

\bibliographystyle{plainnat}
\bibliography{litdb}

\end{document}
